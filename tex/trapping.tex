%
%\documentclass[useAMS,8pt,twocolumn]{article}
\documentclass[apj]{emulateapj}
\bibliographystyle{apj}

%\pdfpagewidth  200mm
%\pdfpageheight 297mm
%\addtolength{\oddsidemargin}{-.4in}
%%\addtolength{\evensidemargin}{-.4in}                                           
%\addtolength{\textwidth}{1.1in}
%\addtolength{\topmargin}{-.5in}
%\addtolength{\textheight}{1.6in}


%\usepackage[T1]{fontenc}
%\usepackage[latin9]{inputenc}
%\usepackage{geometry}
%\geometry{verbose,letterpaper,tmargin=2cm,bmargin=2cm,lmargin=2cm,rmargin=2cm}
\usepackage{apjfonts}
\usepackage{graphicx}
\usepackage{color}
\usepackage{amsbsy}
%\usepackage{mathrsfs}
\usepackage{mathpazo,bm}
\usepackage{amsmath}
\usepackage{multirow}
\usepackage{natbib}
%\usepackage{amssymb}


\newlength{\hfwidth}
\newlength{\hfwidthsingle}
\addtolength{\hfwidthsingle}{.5\textwidth} %single fig in figure
\addtolength{\hfwidth}{.497\textwidth} %single fig in figure*

\newcommand{\pderiv}[2]{\frac{\partial #1}{\partial #2}}
\newcommand{\pderivn}[3]{\frac{\partial^{#3} #1}{\partial #2^{#3}}}
\newcommand{\vt}[1]{\mathbf{#1}}       %for tensors
\renewcommand{\v}[1]{{\boldsymbol{#1}}} %for vectors
\newcommand{\Ro}{\mathrm{Ro}}
\def\white#1{\textcolor{white}{#1}}
\definecolor{brown}{rgb}{0.42,0.24,0.07}
\def\brown#1{\textcolor{brown}{#1}}
\newcommand{\comm}[1]{({\it \brown{#1}})}
\newcommand{\del}{\v{\nabla}}
\newcommand{\grad}{\del}
\newcommand{\Div}{\del\cdot}
\newcommand{\Divp}[1]{\del\cdot\left(#1\right)}
\newcommand{\curl}{\del\times}
\newcommand{\Laplace}{\nabla^2}
\newcommand{\de}{\mathrm{d}} 
\newcommand{\rint}{$s_{\rm int}\,$}
\newcommand{\rext}{$s_{\rm ext}\,$}
\newcommand{\hats}{\hat{\v{s}}}
\newcommand{\hatphi}{\hat{\v{\phi}}}
\newcommand{\hatz}{\hat{\v{z}}}
\newcommand{\hatx}{\hat{\v{x}}}
\newcommand{\haty}{\hat{\v{y}}}
\newcommand{\hatm}{{\hat{m}}}
\newcommand{\hatn}{{\hat{n}}}
\newcommand{\he}{{\hat{e}}}
\newcommand{\hmu}{{\hat{\mu}}}
\newcommand{\hnu}{{\hat{\nu}}}
\newcommand{\va}{v_{_{\rm A}}}
\newcommand{\eps}{\epsilon}
\newcommand{\cte}{\textrm{cte}}
\newcommand{\up}{u^\prime}
\newcommand{\pp}{p^\prime}

\newcommand{\Eq}[1]{Eq. (\ref{#1})}
\newcommand{\Eqs}[2]{Eqs. (\ref{#1}) and~(\ref{#2})}
\newcommand{\Eqss}[2]{Eqs. (\ref{#1})--(\ref{#2})}
\newcommand{\eq}[1]{\Eq{#1}}
\newcommand{\eqs}[2]{\Eqs{#1}{#2}}
\newcommand{\eqss}[2]{\Eqss{#1}{#2}}
\newcommand{\eqp}[1]{(Eq. \ref{#1})}

\newcommand{\App}[1]{Appendix~\ref{#1}}
\newcommand{\app}[1]{\App{#1}}
\newcommand{\Figure}[1]{Figure~\ref{#1}}
\newcommand{\Fig}[1]{Fig.~\ref{#1}}
\newcommand{\fig}[1]{\Fig{#1}}
\newcommand{\Figs}[2]{Figs.~\ref{#1} and \ref{#2}}

\newcommand{\beq}{\begin{equation}}
\newcommand{\eeq}{\end{equation}}
\newcommand{\beqn}{\begin{eqnarray}}
\newcommand{\eeqn}{\end{eqnarray}}
\newcommand{\epsp}{\varepsilon_{_{+}}}
\newcommand{\epsm}{\varepsilon_{_{-}}}

\shorttitle{Dust in disk vortices}
\shortauthors{Lyra \& Lin}

\slugcomment{Draft version}

\begin{document}
%\date{}

\title{Steady state of dust distributions in disk vortices}
\author{Wladimir Lyra\altaffilmark{1,2,3,$\star$} and Min-Kai Lin\altaffilmark{4,$\star$}}
\email{wlyra@caltech.edu, \ mklin924@cita.utoronto.ca}
\altaffiltext{1}{Jet Propulsion Laboratory, California Institute of Technology, 4800 Oak Grove Drive, Pasadena, CA, 91109, USA}
\altaffiltext{2}{Division of Geological \& Planetary Sciences, California Institute of Technology, 1200 E California Blvd MC 150-21, Pasadena, CA 91125 USA}
\altaffiltext{3}{Sagan Fellow}
\altaffiltext{4}{Canadian Institute for Theoretical Astrophysics , 60
  St. George Street, Toronto, Ontario, M5S 3H8, Canada}
\altaffiltext{$\star$}{Both authors contributed equally to this work}

\begin{abstract}
.
\end{abstract}

\section{Introduction}
\label{sect:introduction}

ALMA is returning images of transitional disks in which large
asymmetries are seen in the dust distribution of mm-size in the outer
disk. The explanation in vogue borrows from the protoplanetary disk 
literature by suggesting that these asymmetries are the
result of dust trapping in giant vortices. The vortex excitation method would
be Rossby Wave Instability (RWI) at planetary gap edges. 

Due to the drag force, dust trapped in vortices will accumulate in the
center. Diffusion is needed to maintain a steady state over the
lifetime of the disk. In here, we seek to obtain steady state solution
of the diffusion-advection equation of dust in the possible vortices 
seen in the observations. 

\section{Dust steady state} 
\label{sect:model-equations}

Considering the dust is of small sizes, we can treat it as a
fluid. The dust should then follow the continuity equation 

\beq
  \pderiv{\rho_d}{t} = -(\v{v}\cdot\del)\rho_d - \rho_d \Div{\v{v}} + D\Laplace{\rho_d}
\eeq

\noindent where $\rho_d$ is the dust density, $\v{v}$ is the dust
velocity, and $D$ is the diffusion coefficient (the diffusion is due
to elliptical turbulence in the vortex core and in general will be
different than the turbulent viscosity in the disk). 

To derive the velocities, instead of solving the momentum equations
for the dust, we make use of the relative velocity, following Youdin
(2008)  

\beq
\v{v} = \v{u} + \tau  c_s^2 \,  \grad{\ln \rho_g} 
\label{eq:v}
\eeq

\noindent valid for isothermal gas and small values of the friction time $\tau$. The
velocity $\v{u}$ is the velocity of the gas. Inside a vortex, it is
divergenceless and must satisfy 

\beq
  u_x = \varOmega_V y / \chi \qquad  u_y= -\varOmega_V x \chi
\eeq

\noindent for a vortex of aspect ratio $\chi= y/x > 1$. The Kida
solution $\varOmega_V = 3\Omega_K/2(\chi-1)$ smoothly matches 
this velocity field to the Keplerian shear. Taking the divergent of
\eq{eq:v} gives 

\beq
\Div{\v{v}} = \tau c_s^2 \, \Laplace{\, \ln \, \rho_g} 
\label{eq:divv}
\eeq

\noindent and we can find the Laplacian of the pressure via the Euler
equation. At steady state, the force balance yields 

\begin{eqnarray}
c_s^2 \pderiv{\ln \,\rho_g}{x} &=& 3\varOmega^2 x + 2\varOmega u_y -
u_y\pderiv{u_x}{y} \nonumber \\
&=& \left(3\varOmega^2 - 2\Omega\Omega_V \chi + \varOmega_V^2\right) x  = -C_1 x  \\
c_s^2 \pderiv{\ln \,\rho_g}{y} &=& - 2\varOmega u_x -
u_x\pderiv{u_y}{x} \nonumber \\
&=& \left(-2\Omega\Omega_V/\chi + \Omega_V^2\right) y = -C_2 y
\end{eqnarray}

\noindent  so, with $\omega_{_V}=\varOmega_V/\varOmega$

\beqn
\Div{\v{v}} &=& -\tau (C_1+C_2) = - C\\
&=& - \tau\varOmega^2 \left[2\omega_{_V}\left(\frac{\chi^2+1}{\chi}\right) - (2\omega_{_V}^2 + 3) \right]
\eeqn

\noindent Where we define $C$ as positive, so that the divergence is negative (physically meaning that the dust gets trapped). Replacing that in the continuity equation, and setting $\partial_t$ =
0 for steady state, 

\beq
D\Laplace{\rho_d} + C\rho_d -  (\v{u}\cdot\del)\rho_d = 0 
\eeq

\noindent where we set $\v{v}=\v{u}$ in the first term, accurate to
first order since $\tau$ is supposed to be small. Substituting the gas velocity,

\beq
   -\Omega_V y / \chi \partial_x \rho_d +  \Omega_V x \chi \partial_y
  \rho_d + C \rho_d  + D\partial_x^2{\rho_d} + D\partial_y^2{\rho_d} = 0.
\eeq

Dividing by $D$ and substituting $A=\varOmega_V/D$ and $B=C/D$, we
arrive at the modified advection-diffusion equation that should
determine the steady-state distribution of the vortex-trapped dust.

\beq
  \pderivn{\rho_d}{x}{2} + \pderivn{\rho_d}{y}{2} - A  y\chi^{-1} \pderiv{\rho_d}{x} +  A x \chi \pderiv{\rho_d}{y} + B \rho_d   = 0 
\eeq


\section{Change of variable}

We change variables to the coordinate system used in Chang \& Oishi (2010)

\beqn
  x &=& a \cos\phi \\
  y &=& a\chi\sin\phi
\eeqn

The system is not orthogonal, but it has the advantage of matching the
aspect ratio of the ellipses (In contrast, the elliptic coordinate
system, though orthogonal, describes a system of confocal ellipses of
different aspect ratio, that does not coincide with the geometry of
the problem). 

In these coordinates, the transformations are $\partial_\v{a} =
\vt{A} \partial_\v{x}$, and $\partial_\v{x} =
\vt{A}^{-1} \partial_\v{a}$, with $\v{a} = (a,\phi)^T$, and $\v{x} =
(x,y)^T$. That is

\beq
\left[\begin{array}{c}
    \pderiv{f}{a}  \\
    \pderiv{f}{\phi}
  \end{array}\right] = \vt{A} 
  \left[\begin{array}{c}
      \pderiv{f}{x}  \\
      \pderiv{f}{y}
    \end{array}\right] 
\eeq

With 

\beq
\vt{A} = \left[\begin{array}{cc}
\pderiv{x}{a}  & \pderiv{y}{a}  \\
\pderiv{x}{\phi}  & \pderiv{y}{\phi} \\
\end{array}\right] = \left[\begin{array}{cc}
\cos\phi  & \chi\sin\phi  \\
-a\sin\phi  & a\chi\cos\phi \\
\end{array}\right] 
\eeq

The inverse matrix is 

\beq
\vt{A}^{-1} = \frac{1}{a\chi} \left[\begin{array}{cc}
a\chi\cos\phi  & -\chi\sin\phi  \\
a\sin\phi  & \cos\phi \\
\end{array}\right]  
\eeq

The transformations are therefore

\beqn
\pderiv{}{x} &=& \cos\phi \pderiv{}{a} - \frac{\sin\phi}{a} \pderiv{}{\phi} \\
\pderiv{}{y} &=& \frac{1}{\chi}\left(\sin\phi \pderiv{}{a} + \frac{\cos\phi}{a} \pderiv{}{\phi} \right)
\eeqn

And the second derivatives are 

\beqn
\pderivn{}{x}{2} &=& \left(\cos\phi \pderiv{}{a} - \frac{\sin\phi}{a}
  \pderiv{}{\phi}\right) \left(\cos\phi \pderiv{}{a} -
  \frac{\sin\phi}{a} \pderiv{}{\phi}\right) \nonumber \\ \\
&=& \cos^2\phi\partial^2_a + a^{-2}\sin^2\phi \partial^2_\phi -
a^{-1}\sin 2\phi \partial^2_{a\phi} \nonumber \\
&&+ a^{-2}\sin 2\phi\partial_\phi +
a^{-1}\sin^2\phi \partial_a 
\eeqn

\beqn
\pderivn{}{y}{2} &=& \frac{1}{\chi^2}\left(\sin\phi \pderiv{}{a} +
  \frac{\cos\phi}{a} \pderiv{}{\phi} \right) \left(\sin\phi
  \pderiv{}{a} + \frac{\cos\phi}{a} \pderiv{}{\phi} \right) \nonumber
\\ \\
&=& \frac{1}{\chi^2}\left[ \sin^2\phi \partial^2_a +
  a^{-2}\cos^2\phi \partial^2_\phi + a^{-1}\sin
  2\phi \partial^2_{a\phi}\right. \nonumber \\
&&\left.- a^{-2}\sin 2\phi \partial_\phi +
  a^{-1}\cos^2\phi \partial_a \right] 
\eeqn

The Laplacian is thus 

\beqn
\Laplace{} &= &\frac{1}{2}\left[ \epsm \cos 2\phi + \epsp\right] \partial^2_a  \\
                &+& \frac{1}{2a^2}\left[ \epsp - \epsm \cos 2\phi\right] \partial^2_\phi  \\
                &-& \frac{\sin 2\phi}{a}\epsm \partial^2_{a\phi}  \\
                &+& \frac{1}{2a}\left[ \epsp - \epsm \cos 2\phi\right] \partial_a  \\
                &+& \frac{\sin 2\phi}{a^2} \epsm\partial_\phi
\eeqn

with $\varepsilon_{\pm} = (1 \pm \chi^{-2})$. 

As for the advection term, $\v{v}\cdot\grad$, with $\v{v}$=$\v{u}$=
$\varOmega_V a(\sin\phi,-\chi\cos\phi)$, it conveniently reduces to  $\v{v}\cdot\grad=  -\varOmega_V \partial_\phi $

The dust-trapping equation is therefore 

\beq \label{eq:dust-trapping-uvzero}
\left(\Laplace{} + A\partial_\phi + B\right) \rho_d = 0 
\eeq

\section{Axisymmetric dust distribution}

For an ``axisymmetric'' vortex, we set $\partial_\phi$ = 0, so 

\beq
\left[\frac{1}{2}\left( \epsm \cos 2\phi +\epsp\right) \partial^2_a   + \frac{1}{2a}\left( \epsp - \epsm\cos 2\phi\right) \partial_a  + B\right] \rho_d = 0 
\eeq

We now integrate the coefficients in $\phi$, from 0 to 2$\pi$, to get
rid of the dependencies on trigonometric functions. This yields

\beq
\frac{\epsp}{2}\pderivn{\rho_d}{a}{2} +
\frac{\epsp}{2a}\pderiv{\rho_d}{a} + B \rho_p = 0  
\eeq

That is, 

\beq
y^{\prime\prime} + \frac{1}{a}y^\prime + k^2 y = 0 
\label{eq:axisymmetric}
\eeq where $y(a) = \rho_d$ and $k^2 = 2B/\epsp$.  The general solution of \eq{eq:axisymmetric} is a
sum of Bessel functions: 

\beq
y(a) = c_1 J_0 (ka) + c_2 Y_0(ka) 
\eeq

Since  $Y_0$ diverges at the origin, we can physically discard 
this solution, setting $c_2=0$. So, the axisymmetric mode is

\beq\label{eq:axi}
\rho(a) = c_1 \ J_0 (ka)
\eeq

Or

\beq
\rho(a) = c_1 \ J_0 \left(a\varOmega \ f(\chi) \sqrt{\frac{2\tau}{D}} \right)
\eeq

\noindent with 

\beqn
f^2(\chi) &=& \epsp^{-1} \left[2\omega_{_V}\left(\frac{\chi^2+1}{\chi}\right) - (2\omega_{_V}^2 + 3) \right]\nonumber \\
          &=& 2\omega_{_V}\chi - \epsp^{-1}(2\omega_{_V}^2 + 3)
\eeqn

We can substitute the diffusion coefficient $D=\delta \varOmega H^2$ where 
$\delta$ is a dimensionless coefficient, and ${\rm St} = \tau\varOmega$ for 
the Stokes number, writing thus 

\beq
\rho(a) = c_1 \ J_0 \left(\frac{a}{H} \ f(\chi) \sqrt{\frac{2{\rm St}}{\delta}}\right)
\eeq 

The Bessel function $J_0(x)$ is
non-motononic, with the first root around $x=2.5$. As the density
cannot be negative, that sets the validity of our solution. The
solution is also constrained by our use of the shearing sheet
equations, so the results should not be valid for $x \gg H$. 
 
We plot the function $f(\chi)$ for the Kida and the Goodman
solution, which are defined in the real axis only for $\chi > 2$ ($\psi^2
< 0$ for less than that). The Goodman solition tends to an asymptote
around 0.7. The Kida solution has a  tail around $0.5\pm0.25$ in the
interval of physical relevance. The first zero is therefore around
$a_0 \approx H \sqrt{\delta/{\rm St}}$.  As we made use of the
approximation $\v{v}=\v{u}$, ${\rm St} \ll 1$, and the solution is indeed $a_0 \gg H$, for finite values of $\delta$.

\subsection{Axisymmetric u $\ne$ v }

In deriving eq 10, we assumed $\v{u}=\v{v}$, for practical reasons. We
relax now this approximation, substituting $\v{v}$ for eq 2. The
advection term then becomes 

\beqn
(\v{v}\cdot\del) &=& (\v{u}\cdot\del)  + \tau c_s^2 (\grad\ln \cdot
\del) \nonumber \\
&=& - \varOmega_V \partial_\phi - \tau \left( C_1 x \partial_x + C_2 y
  \partial_y\right) \nonumber \\
&=& - \varOmega_V \partial_\phi - \tau \left( C_1 \cos^2\phi   + C_2
  \sin^2\phi \right) a \ \partial_a \nonumber \\
&&+ \tau \cos\phi\sin\phi  (C_1 - C_2) \partial_\phi
\eeqn

And, for the axisymmetric mode, $\partial_\phi=0$, so

\beq
(\v{v}\cdot\del) = - \tau a \left( C_1 
  \cos^2\phi  + C_2  \sin^2\phi \right) \partial_a
\eeq

Integrating it over $\frac{1}{2\pi}\int_0^{2\pi} d\phi$, 

\beq
(\v{v}\cdot\del) = - \frac{C a}{2} \pderiv{}{a}
\eeq

And the dust trapping equation becomes 

\beq
\left[\pderivn{}{a}{2} + \left(\frac{1}{a} +
  \frac{k^2a}{2}\right)\pderiv{}{a} + k^2\right]\rho_d = 0 
\eeq

The solution for which is 

\beq
\rho_d(a) = \exp\left(-\frac{k^2a^2}{4}\right)  \left[c_1 + c_2 {\rm
    Ei}\left(\frac{k^2a^2}{4}\right)\right]
\eeq

\noindent where ${\rm Ei}(x)$ is the exponential integral function. 
Since it diverges at the origin, $c_2$ has to be zero, and 

\beq
\rho_d(a) = \rho_{d0} \exp\left(-\frac{a^2}{2H_V^2}\right)
\eeq

with $H_V = \sqrt{2}/k$ for symmetry with the gas sonic scale. We can 
rewrite this length scale recalling that 

\beqn
k^2 = \frac{2C}{\epsp D} = \frac{2{\rm St}}{\delta H^2} f^2(\chi)
\eeqn 

\noindent We can then write 

\beq\label{eq:gen_axi}
\rho_d(a) = \rho_{d0} \exp\left(-\frac{a^2}{2H_a^2}\right)
\eeq
\beq
 H_V = \frac{H}{f(\chi)} \sqrt{\frac{\delta}{\rm St}}
\eeq

\noindent where $\psi(\chi)$ depends on the vortex solution used. We plot 
$\psi(\chi)$ for the Kida and Goodman solutions in \fig{fig:fchi}.


\section{Non-axisymmetric problem}
For the non-axisymmetric problem ($\partial_\phi\neq0$), we first
return to the limit $\tau\to0$ or $\rm{St}\ll 1$ for dust velocity and
formally set $\bm{v}=\bm{u}$ in the advection term 
$\bm{v}\cdot\del$. We are therefore comparing dust advection due to
the background vortex flow and diffusion. 
%This is expected to be valid close to the vortex center
%($a\to0$). Furthermore, Eq. \ref{eq:gen_axi} and
%Eq. \ref{eq:axi} are identical at $O(k^2a^2)$.   

\subsection{Conversion to ordinary differential equations}
The dust density $\rho_d$ is periodic in the $\phi$ co-ordinate. We
therefore seek solutions of the form

\beq\label{eq:series}
\rho_d(a,\phi) = {\rm Re}
\left[\sum_{n=0}^\infty\rho_n(a)\exp{\left(\mathrm{i}n\phi\right)} \right].
\eeq
Henceforth, we will drop the real part notation for convenience. Inserting
\eq{eq:series} into the partial differential equation \eqp{eq:dust-trapping-uvzero},
multiplying by $\exp{(-\mathrm{i}m\phi)}$ and integrating the resulting
expressions over the $\phi$ co-ordinate, we arrive at a set of
ordinary differential equations,

\beqn\label{eq:ode1}
&&0=\epsilon_{-}\left[\frac{1}{4}\rho_{m-2}^{\prime\prime}
-\frac{1}{2a}\left(m-\frac{3}{2}\right)\rho_{m-2}^\prime +
\frac{m(m-2)}{4a^2}\rho_{m-2}\right] \nonumber\\
&&+\frac{\epsilon_+}{2}\left(\rho_m^{\prime\prime}+\frac{1}{a}\rho_m^\prime\right)
-\left(\frac{m^2\epsilon_+}{2a^2}-\mathrm{i}mA-B\right)\rho_m\nonumber\\
&&+\epsilon_{-}\left[\frac{1}{4}\rho_{m+2}^{\prime\prime}
+\frac{1}{2a}\left(m+\frac{3}{2}\right)\rho_{m+2}^\prime +
\frac{m(m+2)}{4a^2}\rho_{m+2} \right]
\eeqn
for each $m$. We only consider even $m$ \comm{why?}. For $m=0$, the $\rho_{m-2}$
terms are absent. We require $\rho_m^\prime(0)=0$ and
$\rho_{m\geq2}(0)=0$ \comm{why?}. 

It will be convenient to write the above equations
in operator form,
\beq\label{eq:ode2}
\tilde{\chi}\mathcal{B}_m\rho_{m-2}(x) + \mathcal{A}_m\rho_m(x) + \tilde{\chi}\mathcal{C}_m\rho_{m+2}(x)=0,
\eeq
where $x\equiv ka$ (not to be confused with the original Cartesian
co-ordinate), $\tilde{\chi}\equiv(\chi^2-1)/[2(\chi^2+1)]$, and 
\beqn\label{eq:ops}
\mathcal{B}_m &\equiv& \frac{d^2}{dx^2} -
\frac{2}{x}\left(m-\frac{3}{2}\right)\frac{d}{dx} + \frac{m(m-2)}{x^2},\\
\mathcal{A}_m &\equiv& \frac{d^2}{dx^2} + \frac{1}{x}\frac{d}{dx} +
\left(k_m^2 - \frac{m^2}{x^2}\right),\\
\mathcal{C}_m  &\equiv& \frac{d^2}{dx^2} +
\frac{2}{x}\left(m+\frac{3}{2}\right)\frac{d}{dx} +
\frac{m(m+2)}{x^2}, 
\eeqn
where $k_m^2 \equiv 1+\mathrm{i}mA/B$. We note that
\beqn\label{eq:ops2}
\mathcal{B}_mJ_{m-2}(\mu x) &=& \mu^2J_m(\mu x),\\
\mathcal{A}_mJ_m(\mu x) &=& \left(k_m^2 - \mu^2\right)J_m(\mu x),\\
\mathcal{C}_mJ_{m+2}(\mu x) &= & = \mu^2J_m(\mu x),
\eeqn
where $\mu$ is a constant. 

\subsection{$x\ll 1$}
As $x\to0$, we expect $\rho_0$ to be the dominant term in
\eq{eq:series}, because the non-axisymmetric components vanish
at the origin. Furthermore, we require $|\rho_m|$ to decrease with
respect to $m$ in order to achieve convergence. In this section we
seek the simplest non-axisymmetric solution
\beq
\rho(a,\phi) \simeq \rho_0(a) + \rho_2(a)\exp{\left(2\mathrm{i}m\phi\right)}. 
\eeq
Specifically, we write
\beq
\rho_0(x) = D_0J_0(k_0x) + \epsilon(x),  
\eeq
where $\epsilon(x)$ represents the correction to the axisymmetric
solution due to $\rho_2(x)$. $D_0$ is a constant. The ODEs to be
solved are 
\beqn
\mathcal{A}_0\epsilon(x) &=& -\tilde{\chi}\mathcal{C}_0\rho_2(x),\label{eq:m2_eq1}\\
\mathcal{A}_2\rho_2(x) &=& -\tilde{\chi}\left[D_0k_0^2J_2(k_0x) +
  \mathcal{B}_2\epsilon(x)\right].\label{eq:m2_eq2} 
\eeqn

At this stage, we make the \emph{assumption} that the $\epsilon$ term
in \eq{eq:m2_eq2} can be neglected \comm{why?}. Thus $\rho_2$ satisfies
\beq\label{eq:rho2_approx}
\mathcal{A}_2\rho_2(x) \simeq -\tilde{\chi}D_0k_0^2J_2(k_0x).
\eeq
A solution to \eq{eq:rho2_approx} is
\beq\label{eq:rho2_approx_sol}
\rho_2(x)  = -\frac{\tilde{\chi}D_0k_0^2J_2(k_0x)}{k_2^2 - k_0^2}. 
\eeq
The general solution to \eq{eq:rho2_approx} would also include the
complementary function $J_2(k_2x)$. However, $k_2$ is complex and $|k_2|\gg 1$ (because we are considering
$\mathrm{St}\ll1$), so $J_2(k_2x)$
may not be well behaved at boundaries. We have therefore taken the
particular integral to \eq{eq:rho2_approx} as the solution for
$\rho_2$. This is to regard $\rho_0$ as a source term for
$\rho_2$. The regime $|k_2|\gg1$ corresponds to  
$\rho_2$ being a small perturbation to the
axisymmetric solution. 

We can use \eq{eq:rho2_approx_sol} in \eq{eq:m2_eq1} to
calculate the correction term $\epsilon$. The ODE to solve is
\beqn
\mathcal{A}_0\epsilon(x)
&=&\frac{\tilde{\chi}^2D_0k_0^2}{k_2^2-k_0^2}\mathcal{C}_0J_2(k_0x) \nonumber \\ 
&=&\frac{\tilde{\chi}^2D_0k_0^4J_0(k_0x)}{k_2^2-k_0^2}.
\eeqn
The particular integral is 
\beq
\epsilon(x) = \frac{\tilde{\chi}^2D_0k_0^3xJ_1(k_0x)}{2\left(k_2^2 -
  k_0^2\right)}. 
\eeq
Using this expression for $\epsilon(x)$, we can evaluate
$\mathcal{B}_2\epsilon(x)$ in order to assess our assumption that
$\epsilon(x)$ has a negligible contribution to $\rho_2$. We find

\beq
\mathcal{B}_2\epsilon(x) =
-\frac{\tilde{\chi}^2D_0k_0^5xJ_1(k_0x)}{2\left(k_2^2 -
  k_0^2\right)}. 
\eeq

Both the $\tilde{\chi}^2$ term and the $k_2^2$ term contributes to
making $\mathcal{B}_2\epsilon$ small in comparison to the $J_2$ term
on the RHS of \eq{eq:m2_eq2}. 

An iteration scheme can now be set up. One can update
$\rho_2$ with the inclusion of $\mathcal{B}_2\epsilon(x)$ using
\eq{eq:m2_eq2}, then in turn update $\epsilon(x)$ using
\eq{eq:m2_eq1} and so on. 


\subsection{Lower $m$ mode as a source for higher $m$ mode}
Let us assume that for each $m\geq2$, $\rho_{m-2}$ acts as a source
for $\rho_{m}$, and that the effect due to $\rho_{m+2}$ can be
neglected. That is
\beq
\mathcal{A}_m\rho_m(x) = -\tilde{\chi}\mathcal{B}_m\rho_{m-2}. 
\eeq 
Then, by induction,
\beq
\rho_m(x) =D_0
\frac{(-1)^{m/2}\tilde{\chi}^{m/2}k_0^m}{\prod_{l=1}^{m/2}(k_{2l}^2-k_0^2)}J_m(k_0x). 
\eeq

%\begin{figure}
%  \begin{center}
%    \resizebox{\columnwidth}{!}{\includegraphics{../figs/damped_oscl.png}}
%  \end{center}
%\caption[]{Two of the five solutions of \eq{eq:polynomium}, corresponding
%to damped oscillations through most of the parameter space. In a small region
%(high dust-to-gas ratio and high frequency) modes are exponentially damped
%without oscillating. The other three solutions \fig{fig:growth} involve linear
%growth.}
% \label{fig:damped-oscl}
%\end{figure}

\section{Conclusions}

\acknowledgments

\begin{thebibliography}{}
\expandafter\ifx\csname natexlab\endcsname\relax\def\natexlab#1{#1}\fi
\end{thebibliography}

\end{document}
