%
\documentclass[useAMS,8pt,twocolumn]{article}
%\bibliographystyle{apj}

\pdfpagewidth  200mm
\pdfpageheight 297mm
\addtolength{\oddsidemargin}{-.4in}
%\addtolength{\evensidemargin}{-.4in}                                           
\addtolength{\textwidth}{1.1in}
\addtolength{\topmargin}{-.5in}
\addtolength{\textheight}{1.6in}


%\usepackage[T1]{fontenc}
%\usepackage[latin9]{inputenc}
%\usepackage{geometry}
%\geometry{verbose,letterpaper,tmargin=2cm,bmargin=2cm,lmargin=2cm,rmargin=2cm}
%\usepackage{apjfonts}
\usepackage{graphicx}
\usepackage{color}
\usepackage{amsbsy}
%\usepackage{mathrsfs}
\usepackage{mathpazo,bm}
\usepackage{amsmath}
\usepackage{multirow}
\usepackage{natbib}
%\usepackage{amssymb}


\newlength{\hfwidth}
\newlength{\hfwidthsingle}
\addtolength{\hfwidthsingle}{.5\textwidth} %single fig in figure
\addtolength{\hfwidth}{.497\textwidth} %single fig in figure*

\newcommand{\pderiv}[2]{\frac{\partial #1}{\partial #2}}
\newcommand{\pderivn}[3]{\frac{\partial^{#3} #1}{\partial #2^{#3}}}
\newcommand{\vt}[1]{\mathbf{#1}}       %for tensors
\renewcommand{\v}[1]{{\boldsymbol{#1}}} %for vectors
\newcommand{\Ro}{\mathrm{Ro}}
\def\white#1{\textcolor{white}{#1}}
\newcommand{\del}{\v{\nabla}}
\newcommand{\grad}{\del}
\newcommand{\Div}{\del\cdot}
\newcommand{\Divp}[1]{\del\cdot\left(#1\right)}
\newcommand{\curl}{\del\times}
\newcommand{\Laplace}{\nabla^2}
\newcommand{\de}{\mathrm{d}} 
\newcommand{\rint}{$s_{\rm int}\,$}
\newcommand{\rext}{$s_{\rm ext}\,$}
\newcommand{\hats}{\hat{\v{s}}}
\newcommand{\hatphi}{\hat{\v{\phi}}}
\newcommand{\hatz}{\hat{\v{z}}}
\newcommand{\hatx}{\hat{\v{x}}}
\newcommand{\haty}{\hat{\v{y}}}
\newcommand{\hatm}{{\hat{m}}}
\newcommand{\hatn}{{\hat{n}}}
\newcommand{\he}{{\hat{e}}}
\newcommand{\hmu}{{\hat{\mu}}}
\newcommand{\hnu}{{\hat{\nu}}}
\newcommand{\va}{v_{_{\rm A}}}
\newcommand{\eps}{\epsilon}
\newcommand{\cte}{\textrm{cte}}
\newcommand{\up}{u^\prime}
\newcommand{\pp}{p^\prime}

\newcommand{\Eq}[1]{Eq. (\ref{#1})}
\newcommand{\Eqs}[2]{Eqs. (\ref{#1}) and~(\ref{#2})}
\newcommand{\Eqss}[2]{Eqs. (\ref{#1})--(\ref{#2})}
\newcommand{\eq}[1]{\Eq{#1}}
\newcommand{\eqs}[2]{\Eqs{#1}{#2}}
\newcommand{\eqss}[2]{\Eqss{#1}{#2}}
\newcommand{\App}[1]{Appendix~\ref{#1}}
\newcommand{\app}[1]{\App{#1}}
\newcommand{\Figure}[1]{Figure~\ref{#1}}
\newcommand{\Fig}[1]{Fig.~\ref{#1}}
\newcommand{\fig}[1]{\Fig{#1}}
\newcommand{\Figs}[2]{Figs.~\ref{#1} and \ref{#2}}

\newcommand{\beq}{\begin{equation}}
\newcommand{\eeq}{\end{equation}}
\newcommand{\beqn}{\begin{eqnarray}}
\newcommand{\eeqn}{\end{eqnarray}}
\newcommand{\epsp}{\varepsilon_{_{+}}}
\newcommand{\epsm}{\varepsilon_{_{-}}}

\begin{document}
\date{}

\title{Steady-state of dust distributions in vortices}
\author{}

\maketitle

ALMA is returning images of transitional disks in which large
asymmetries are seen in the dust distribution of mm-size in the outer
disk. The explanation in vogue borrows from the protoplanetary disk 
literature by suggesting that these asymmetries are the
result of dust trapping in giant vortices. The vortex excitation method would
be Rossby Wave Instability (RWI) at planetary gap edges. 

Due to the drag force, dust trapped in vortices will accumulate in the
center. Diffusion is needed to maintain a steady state over the
lifetime of the disk. In here, we seek to obtain steady state solution
of the diffusion-advection equation of dust in the possible vortices 
seen in the observations. 

\section{Dust steady state} 

Considering the dust is of small sizes, we can treat it as a
fluid. The dust should then follow the continuity equation 

\beq
  \pderiv{\rho_d}{t} = -(\v{v}\cdot\del)\rho_d - \rho_d \Div{\v{v}} + D\Laplace{\rho_d}
\eeq

\noindent where $\rho_d$ is the dust density, $\v{v}$ is the dust
velocity, and $D$ is the diffusion coefficient (the diffusion is due
to elliptical turbulence in the vortex core and in general will be
different than the turbulent viscosity in the disk). 

To derive the velocities, instead of solving the momentum equations
for the dust, we make use of the relative velocity, following Youdin
(2008)  

\beq
\v{v} = \v{u} + \tau  c_s^2 \,  \grad{\ln \rho_g} 
\label{eq:v}
\eeq

\noindent valid for isothermal gas and small values of the friction time $\tau$. The
velocity $\v{u}$ is the velocity of the gas. Inside a vortex, it is
divergenceless and must satisfy 

\beq
  u_x = \varOmega_V y / \chi \qquad  u_y= -\varOmega_V x \chi
\eeq

\noindent for a vortex of aspect ratio $\chi= y/x > 1$. The Kida
solution $\varOmega_V = 3\Omega_K/2(\chi-1)$ smoothly matches 
this velocity field to the Keplerian shear. Taking the divergent of
\eq{eq:v} gives 

\beq
\Div{\v{v}} = \tau c_s^2 \, \Laplace{\, \ln \, \rho_g} 
\label{eq:divv}
\eeq

\noindent and we can find the Laplacian of the pressure via the Euler
equation. At steady state, the force balance yields 

\begin{eqnarray}
c_s^2 \pderiv{\ln \,\rho_g}{x} &=& 3\varOmega^2 x + 2\varOmega u_y -
u_y\pderiv{u_x}{y} \nonumber \\
&=& \left(3\varOmega^2 - 2\Omega\Omega_V \chi + \varOmega_V^2\right) x  = -C_1 x  \\
c_s^2 \pderiv{\ln \,\rho_g}{y} &=& - 2\varOmega u_x -
u_x\pderiv{u_y}{x} \nonumber \\
&=& \left(-2\Omega\Omega_V/\chi + \Omega_V^2\right) y = -C_2 y
\end{eqnarray}

\noindent  so, with $\omega_{_V}=\varOmega_V/\varOmega$

\beqn
\Div{\v{v}} &=& -\tau (C_1+C_2) = - C\\
&=& - \tau\varOmega^2\left[ 2\omega_{_V}
  \left(\frac{\chi^2+1}{\chi^2}\right) - \omega_{_V}^2 - 3\right]
\eeqn

\noindent Where we define C as positive, so that the divergence is negative (physically meaning that the dust gets trapped). Replacing that in the continuity equation, and setting $\partial_t$ =
0 for steady state, 

\beq
D\Laplace{\rho_d} + C\rho_d -  (\v{u}\cdot\del)\rho_d = 0 
\eeq

\noindent where we set $\v{v}=\v{u}$ in the first term, accurate to
first order since $\tau$ is supposed to be small. Substituting the gas velocity,

\beq
   -\Omega_V y / \chi \partial_x \rho_d +  \Omega_V x \chi \partial_y
  \rho_d + C \rho_d  + D\partial_x^2{\rho_d} + D\partial_y^2{\rho_d} = 0.
\eeq

Dividing by $D$ and substituting $A=\varOmega_V/D$ and $B=C/D$, we
arrive at the modified advection-diffusion equation that should
determine the steady-state distribution of the vortex-trapped dust.

\beq
  \pderivn{\rho_d}{x}{2} + \pderivn{\rho_d}{y}{2} - A  y\chi^{-1} \pderiv{\rho_d}{x} +  A x \chi \pderiv{\rho_d}{y} + B \rho_d   = 0 
\eeq


\section{Change of variable}

We change variables to the coordinate system used in Chang \& Oishi (2010)

\beqn
  x &=& a \cos\phi \\
  y &=& a\chi\sin\phi
\eeqn

The system is not orthogonal, but it has the advantage of matching the
aspect ratio of the ellipses (In contrast, the elliptic coordinate
system, though orthogonal, describes a system of confocal ellipses of
different aspect ratio, that does not coincide with the geometry of
the problem). 

In these coordinates, the transformations are $\partial_\v{a} =
\vt{A} \partial_\v{x}$, and $\partial_\v{x} =
\vt{A}^{-1} \partial_\v{a}$, with $\v{a} = (a,\phi)^T$, and $\v{x} =
(x,y)^T$. That is

\beq
\left[\begin{array}{c}
    \pderiv{f}{a}  \\
    \pderiv{f}{\phi}
  \end{array}\right] = \vt{A} 
  \left[\begin{array}{c}
      \pderiv{f}{x}  \\
      \pderiv{f}{y}
    \end{array}\right] 
\eeq

With 

\beq
\vt{A} = \left[\begin{array}{cc}
\pderiv{x}{a}  & \pderiv{y}{a}  \\
\pderiv{x}{\phi}  & \pderiv{y}{\phi} \\
\end{array}\right] = \left[\begin{array}{cc}
\cos\phi  & \chi\sin\phi  \\
-a\sin\phi  & a\chi\cos\phi \\
\end{array}\right] 
\eeq

The inverse matrix is 

\beq
\vt{A}^{-1} = \frac{1}{a\chi} \left[\begin{array}{cc}
a\chi\cos\phi  & -\chi\sin\phi  \\
a\sin\phi  & \cos\phi \\
\end{array}\right]  
\eeq

The transformations are therefore

\beqn
\pderiv{}{x} &=& \cos\phi \pderiv{}{a} - \frac{\sin\phi}{a} \pderiv{}{\phi} \\
\pderiv{}{y} &=& \frac{1}{\chi}\left(\sin\phi \pderiv{}{a} + \frac{\cos\phi}{a} \pderiv{}{\phi} \right)
\eeqn

And the second derivatives are 

\beqn
\pderivn{}{x}{2} &=& \left(\cos\phi \pderiv{}{a} - \frac{\sin\phi}{a}
  \pderiv{}{\phi}\right) \left(\cos\phi \pderiv{}{a} -
  \frac{\sin\phi}{a} \pderiv{}{\phi}\right) \nonumber \\ \\
&=& \cos^2\phi\partial^2_a + a^{-2}\sin^2\phi \partial^2_\phi -
a^{-1}\sin 2\phi \partial^2_{a\phi} \nonumber \\
&&+ a^{-2}\sin 2\phi\partial_\phi +
a^{-1}\sin^2\phi \partial_a 
\eeqn

\beqn
\pderivn{}{y}{2} &=& \frac{1}{\chi^2}\left(\sin\phi \pderiv{}{a} +
  \frac{\cos\phi}{a} \pderiv{}{\phi} \right) \left(\sin\phi
  \pderiv{}{a} + \frac{\cos\phi}{a} \pderiv{}{\phi} \right) \nonumber
\\ \\
&=& \frac{1}{\chi^2}\left[ \sin^2\phi \partial^2_a +
  a^{-2}\cos^2\phi \partial^2_\phi + a^{-1}\sin
  2\phi \partial^2_{a\phi}\right. \nonumber \\
&&\left.- a^{-2}\sin 2\phi \partial_\phi +
  a^{-1}\cos^2\phi \partial_a \right] 
\eeqn

The Laplacian is thus 

\beqn
\Laplace{} &= &\frac{1}{2}\left[ \epsm \cos 2\phi + \epsp\right] \partial^2_a  \\
                &+& \frac{1}{2a^2}\left[ \epsp - \epsm \cos 2\phi\right] \partial^2_\phi  \\
                &+& \frac{\sin 2\phi}{a}\epsm \partial^2_{a\phi}  \\
                &+& \frac{1}{2a}\left[ \epsp - \epsm \cos 2\phi\right] \partial_a  \\
                &+& \frac{\sin 2\phi}{a^2} \epsm\partial_\phi
\eeqn

with $\varepsilon_{\pm} = (1 \pm \chi^{-2})$. 

As for the advection term, $\v{v}\cdot\grad$, with $\v{v}$=$\v{u}$=
$\varOmega_V a(\sin\phi,-\chi\cos\phi)$, it conveniently reduces to  $\v{v}\cdot\grad=  -\varOmega_V \partial_\phi $

The dust-trapping equation is therefore 

\beq
\left(\Laplace{} + A\partial_\phi + B\right) \rho_d = 0 
\eeq

For an ``axisymmetric'' vortex, we set $\partial_\phi$ = 0, so 

\beq
\left[\frac{1}{2}\left( \epsm \cos 2\phi +\epsp\right) \partial^2_a   + \frac{1}{2a}\left( \epsp - \epsm\cos 2\phi\right) \partial_a  + B\right] \rho_d = 0 
\eeq

We now integrate the coefficients in $\phi$, from 0 to 2$\pi$, to get
rid of the dependencies on trigonometric functions. This yields

\beq
\frac{\epsp}{2}\pderivn{\rho_d}{a}{2} +
\frac{\epsp}{2a}\pderiv{\rho_d}{a} + B \rho_p = 0  
\eeq

That is, 

\beq
y^{\prime\prime} + \frac{1}{a}y^\prime + k^2 y = 0 
\label{eq:axisymmetric}
\eeq where $y(a) = \rho_d$ and $k^2 = 2B/\epsp$.  The general solution of \eq{eq:axisymmetric} is a
sum of Bessel functions: 

\beq
y(a) = c_1 J_0 (ka) + c_2 Y_0(ka) 
\eeq

Since  $Y_0(0)=0$, we can physically discard this solution, setting
$c_2=0$. So, the axisymmetric mode is

\beq
\rho(a) = c_1 \ J_0 (ka)
\eeq

Or

\beq
\rho(a) = c_1 \ J_0 \left(a\varOmega\sqrt{\frac{2\tau}{D\epsp}} \ f(\chi)\right)
\eeq

with $f^2(\chi) = (2\omega_{_V} (\chi + \chi^{-1}) - \omega_{_V} - 3)$. We 
can substitute the diffusion coefficient $D=\delta \varOmega H^2$ where 
$\delta$ is a dimensionless coefficient, and ${\rm St} = \tau\varOmega$ for 
the Stokes number, writing thus 

\beq
\rho(a) = c_1 \ J_0 \left(\frac{a}{H} \sqrt{\frac{2{\rm St}}{\delta}}\psi(\chi)\right)
\eeq 

with $\psi(\chi) = f(\chi) \epsp^{-1/2}$. The Bessel function $J_0(x)$ is
non-motononic, with the first root around $x=2.5$. As the density
cannot be negative, that sets the validity of our solution. The
solution is also constrained by our use of the shearing sheet
equations, so the results should not be valid for $x \gg H$. 
 
We plot the function $\psi(\chi)$ for the Kida and the Goodman
solution, which are defined in the real axis only for $\chi > 2$ ($\psi^2
< 0$ for less than that). The Goodman solition tends to an asymptote
around 0.7. The Kida solution has a  tail around $0.5\pm0.25$ in the
interval of physical relevance. The first zero is therefore around
$a_0 \approx H \sqrt{\delta/{\rm St}}$.  As we made use of the
approximation $\v{v}=\v{u}$, ${\rm St} \ll 1$, and the solution is indeed $a_0 \gg
H$

\subsection{Axisymmetric u $\ne$ v }

In deriving eq 10, we assumed $\v{u}=\v{v}$, for practical reasons. We
relax now this approximation, substituting $\v{v}$ for eq 2. The
advection term then becomes 

\beqn
(\v{v}\cdot\del) &=& (\v{u}\cdot\del)  + \tau c_s^2 (\grad\ln \cdot
\del) \nonumber \\
&=& - \varOmega_V \partial_\phi - \tau \left( C_1 x \partial_x + C_2 y
  \partial_y\right) \nonumber \\
&=& - \varOmega_V \partial_\phi - \tau \left( C_1 \cos^2\phi   + C_2
  \sin^2\phi \right) a \ \partial_a \nonumber \\
&&+ \tau \cos\phi\sin\phi  (C_1 - C_2) \partial_\phi
\eeqn

And, for the axisymmetric mode, $\partial_\phi=0$, so

\beq
(\v{v}\cdot\del) = - \tau a \left( C_1 
  \cos^2\phi  + C_2  \sin^2\phi \right) \partial_a
\eeq

Integrating it over $\frac{1}{2\pi}\int_0^{2\pi} d\phi$, 

\beq
(\v{v}\cdot\del) = - \frac{C a}{2} \pderiv{}{a}
\eeq

And the dust trapping equation becomes 

\beq
\left[\pderivn{}{a}{2} + \left(\frac{1}{a} +
  \frac{k^2a}{2}\right)\pderiv{}{a} + k^2\right]\rho_d = 0 
\eeq

The solution for which is 

\beq
\rho_d(a) = \exp\left(-\frac{k^2a^2}{4}\right)  \left[c_1 + c_2 {\rm
    Ei}\left(\frac{k^2a^2}{4}\right)\right]
\eeq

where ${\rm Ei}(x)$ is the exponential integral function. Since it diverges
for $x\rightarrow\infty$, $c_2$ has to be zero, and 

\beq
\rho_d(a) = \rho_{d0} \exp\left(-\frac{a^2}{2H_a^2}\right)
\eeq

with $H_a = \sqrt{2}/k$. 

\end{document}
